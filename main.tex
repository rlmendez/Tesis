%%%%%%%%%%%%%%%%%%%%%%%%%%%%%%%%%%%%%%%%%%%%%%%%%%%%%%%%%%%%%%%%%%%%%%%%%%%
%
% Plantilla para un artculo en LaTeX en espaol.
%
%%%%%%%%%%%%%%%%%%%%%%%%%%%%%%%%%%%%%%%%%%%%%%%%%%%%%%%%%%%%%%%%%%%%%%%%%%%

\documentclass[a4paper, 11pt, oneside]{article}


% idioma
\usepackage[utf8]{inputenc}
\usepackage[spanish]{babel}
\usepackage{enumerate}
\usepackage{multirow} % para las tablas

%tablas
\usepackage{booktabs}

%rotar tablas
\usepackage{rotating}

%color tablas
\usepackage{colortbl}

%espaciado
\usepackage{setspace}
\onehalfspacing
\setlength{\parindent}{0pt}
\setlength{\parskip}{2.0ex plus0.5ex minus0.2ex}


%margenes según n. icontec
\usepackage{vmargin}
\setmarginsrb           { 4.0cm}  % left margin
                        { 4.0cm}  % top margcm
                        { 2.0cm}  % right margcm
                        { 3.0cm}  % bottom margcm
                        {  10pt}  % head height
                        {0.25cm}  % head sep
                        {   9pt}  % foot height
                        { 0.3cm}  % foot sep


% inserción url's notas de pie.
\usepackage{url}


% Paquetes de la AMS:
\usepackage{amsmath, amsthm, amsfonts}

% Paquete para resaltar texto con una caja amarilla para correcciones
\usepackage{color}
\newcommand{\hilight}[1]{\colorbox{yellow}{#1}}

% Teoremas
%--------------------------------------------------------------------------
\newtheorem{thm}{Teorema}[section]
\newtheorem{cor}[thm]{Corolario}
\newtheorem{lem}[thm]{Lema}
\newtheorem{prop}[thm]{Proposición}
\theoremstyle{definition}
\newtheorem{defn}[thm]{Definición}
\theoremstyle{remark}
\newtheorem{rem}[thm]{Observación}

% Atajos.
% Se pueden definir comandos nuevos para acortar cosas que se usan
% frecuentemente. Como ejemplo, aqu se definen la R y la Z dobles que
% suelen representar a los conjuntos de nmeros reales y enteros.
%--------------------------------------------------------------------------

\def\RR{\mathbb{R}}
\def\ZZ{\mathbb{Z}}

% De la misma forma se pueden definir comandos con argumentos. Por
% ejemplo, aqu definimos un comando para escribir el valor absoluto
% de algo ms fcilmente.
%--------------------------------------------------------------------------
\newcommand{\abs}[1]{\left\vert#1\right\vert}

% Operadores.
% Los operadores nuevos deben definirse como tales para que aparezcan
% correctamente. Como ejemplo definimos en jacobiano:
%--------------------------------------------------------------------------
\DeclareMathOperator{\Jac}{Jac}



\newcommand\portada{
\begin{titlepage}
		\begin{center}
			{\large \bf Relación de enfermedades respiratorias con la contaminación del                aire bogotano a través de una arquitectura multi - partita computacional}
			\vfill
 			{\large\bf PRESENTADO POR: \par}
			{\large\bf Marco Antonio Méndez \par}
            {\large\bf Hoffman Antonio Márquez}
			\vfill
			{\large\bf UNIVERSIDAD ANTONIO NARIÑO  \par}
			{\large\bf FACULTAD DE INGENIERÍA DE SISTEMAS \par}
			{\large\bf INGENIERIA DE SISTEMAS Y COMPUTACIÓN \par}
			{\large\bf BOGOTÁ D.C.\par}
			{\large\bf MAYO 2016 \par}
		\end{center}
\end{titlepage}
}

\newcommand\contraportada{
	\begin{titlepage}
		\begin{center}
			{\large \bf Relación de enfermedades respiratorias con la contaminación del                aire bogotano a través de una arquitectura multi - partita computacional  } 
			\vfill
 			{\large\bf PRESENTADO POR: \par}
			{\large\bf Marco Antonio Méndez \par}
            {\large\bf Hoffman Antonio Márquez}
			\vfill
			{\large\bf Anteproyecto de grado \par}
			\vfill
			{\large\bf Director: PhD. David Alberto Herrera Alvarez / PhD. Raúl Ernesto Menéndez Mora 
\par}
			\vfill
			{\large\bf UNIVERSIDAD ANTONIO NARIÑO \par}
			{\large\bf FACULTAD DE INGENIERÍA \par}
			{\large\bf INGENIERIA DE SISTEMAS Y COMPUTACIÓN \par}
			{\large\bf BOGOTÁ D.C.\par}
			{\large\bf MAYO 2016 \par}
		\end{center}
\end{titlepage}
}


%--------------------------------------------------------------------------
% \title{\LARGE \bf Plantilla para un artculo \LaTeX }
% % \vfill
% \author{El autor va aqu\\
%   \small Dept. Plantillas y Editores\\
%   \small E12345\\
%   \small Espaa
% }


\begin{document}
\portada
\contraportada
 

% \abstract{Esto es una plantilla simóíéññññple para un artculo en \LaTeX.}



\renewcommand\contentsname{\centering Tabla de Contenidos}
\tableofcontents
\clearpage

\begin{center}
 \section{Resumen}
 \begin{flushleft}
 
 HABLAR DEL OBJETIVO DE MANERA INDIRECTA
 
 El proyecto busca principalmente determinar por medio de información suministrada factores o eventos causantes de las diferentes enfermedades generadas por la contaminación en el aire  
 
 la muestra sera dada por los datasets...
 
 Los datasets seran el punto de partida para el analisis de la información. 
 
 despues el ¿como? 
 
 q como con q

REVISAR...!!!

MAS BIEN ESTE PARRAFO COMO INTRODUCCION
Bogotá, una de las grandes ciudades latinas de gran importancia, no es ajena a la contaminación atmosférica, pues como gran ciudad industrializada el nivel de polución es alta, sumado al gran número de automóviles que se movilizan diariamente en la capital, se comienza a ver un impacto negativo en la salud respiratoria de los capitalinos. De esta manera la información suministrada en los datasets del Hospital del Sur (RIPS Registro Individual de Prestación de Servicios de Salud),  la Red de Monitoreo de Calidad del Aire de Bogotá – RMCAB (Secretaría de Medio Ambiente de Bogotá) y SIVIGILA (Sistema Nacional de Vigilancia en Salud Pública) con base a los datasets proporcionados por las diferentes instituciones publicas se pretende encontrar una relación directa entre los datos para estudios posteriores de carácter medico y ambiental. 

De esta manera la vinculación de los datasets aplicando una arquitectura flexible (Multipartita computacional)que permitira realizar minería de datos, el estudio de los mismos sin importar el origen se protegerá la privacidad, integridad y confidencialidad que se requiera para asegurar un análisis con resultados comprometedores para entender la relación de la mortalidad y hospitalizaciones causadas por Infecciones Respiratorias Agudas posible por orígenes naturales o causadas por el accionar del hombre en la ciudad capitalina.
  

  
 \end{flushleft}
 \clearpage
 
 \section{Introducción}
 \begin{flushleft}
 1. DEFINIR BASICAMENTE QUE SON LAS IRA
 2. EL PROBLEMA DE POR QUE RELACIONAR
 3LAS IRA CON LA CONTAMINACIÓN. 
 3. NOMBRAR DATASETS
 4. LA MANERA DE COMUNICACIÓN
 
 
 MAS RESUMIDO Y NO TAN PRECISO EN INFORMACION
 

 La contaminación del ambiente ha causado varios problemas para la salud humana, en este caso nuestro país no es ajeno al cambio climático y menos en la ciudad de Bogotá, la cual sufre un alto nivel de polución causado por las grandes industrias y el gran número de vehículos. El aire bogotano, a pesar de los controles de las entidades ambientales, no es apto para “respirar” ya que un indicio son los altos casos  de enfermedades respiratorias registradas. El desarrollo del proyecto es contar con tres datasets de diferentes orígenes y de esta manera relacionar los datos de manera “anónima” entre las fuentes, esperando resultados apropiados con respecto al estudio de datos del aire bogotano y registros médicos.
 
 
 OPC PARA DESCRIPCION DEL PROBLEMA
 
Los datasets a estudiar serán suministrados por medio del Hospital del Sur (RIPS Registro Individual de Prestación de Servicios de Salud), al igual que la Red de Monitoreo de Calidad del Aire de Bogotá – RMCAB (Secretaría de Medio Ambiente de Bogotá) y SIVIGILA (Sistema Nacional de Vigilancia en Salud Pública). Los RIPS son el conjunto de datos mínimos que el Sistema General de Seguridad Social en Salud requiere para los procesos de dirección, regulación y control para el servicio de prestación de salud, donde se puede obtener perfiles epidemiológicos, frecuencia de usos y costos de servicios, demanda atendida; para el caso de la RMCAB es el mecanismo que  permite recolectar información sobre la concentración de contaminantes de origen antropogénico y natural con el comportamiento de las variables meteorológicas que regulan la distribución de los mismos en la atmósfera bogotana. El tercer dataset es el SIVIGILA  que realiza una provisión en forma sistemática y oportuna de información sobre la dinámica de los eventos que afecten o puedan afectar la salud de la población Colombiana.

Con el proceso ETL aplicado a los datasets de la investigación, se filtraran los datos relevantes para realizar la minería de datos entre ellas, pero antes de este proceso se realizó un diseño de una arquitectura multi-partita computacional flexible para realizar el datamining.

\end{flushleft}
\clearpage

 \section{Planteamiento del Problema}
\end{center}
\subsection{Descripción del Problema}
En Colombia la tasa de mortalidad en Infecciones Respiratorias Agudas (IRA) en menores de 5 años ha sido un problema de salud pública, ya que algunas enfermedades como la bronquiolitis, la bronconeumonía y la neumonía son las causantes de estadísticas desfavorables. Un claro ejemplo fue el aumento desde el año 2013 en 10\% en casos de muerte de infantes por IRA (Programa Nacional de Prevención Manejo y Control de la Infección Respiratoria Aguda 2014) según el Instituto Nacional de  Salud (SIVIGILA). 

párrafo que una el primero con el tercero

Además, no existe un estudio formal entre las posibles relaciones de focos de contaminación atmosférica a partir de la información de la Red de Monitoreo de Calidad del Aire de Bogotá – RMCAB (Secretaría de Medio Ambiente de Bogotá) vinculando dichos datos con datasets de hospitales bogotanos; la posible relación entre los datasets nombrados por el origen independiente de la información y la confidencialidad de los mismos, obliga a realizar una arquitectura computacional que permita el análisis de los datos respetando la ¿integridad?, privacidad y confidencialidad de cada base de datos.

¿existe otra manera de relacionar las enfermedades?
NOTA: PALABRAS DE PRESENTACION

EN FORMULACION ES MAS VIABLE NOMBRAS LA ARQ MULTI PARTITA

\subsection{Formulación del Problema}
¿Existirá una relación directa entre los posibles focos de contaminación atmosférica dadas por la Red de Monitoreo de Calidad del Aire de Bogotá con las enfermedades respiratorias registradas en el Hospital del Sur y el Sistema Nacional de Vigilancia en Salud Pública? De tal manera también ¿ Comó se garantizará la integridad, privacidad y seguridad durante la comunicación entre los datasets?




\subsection{Justificación}
 NOMBRAR EL PROPOSITO DE OPTAR AL TITULO DE ING SISTEMA


Pekín, capital de China deafortunadamente es referente como la ciudad con el aire menos apto para respirar del mundo, lo que ha causado que el gobierno chino emita una alerta roja donde sugirió no hacer actividades al aire libre ('Calidad del aire de Bogotá tuvo niveles similares a los de Pekín' El Tiempo  4 de febrero de 2016). Bogotá en los últimos años por el crecimiento acelerado en la industria automotriz 16\% anual en los últimos años (Industria automotriz en Bogotá sigue creciendo El Espectador 18 de Enero 2012) y de manera paralela a la industria ha causado un aumentó considerable de emisión de gases de efecto al ambiente.

Actualmente y debido a la alta contaminación que se ha producido  durante los últimos años ya se ha recomendado a los capitalinos, que los niños menores de 5 años utilicen tapabocas y / o se encuentre en lugares "limpios" sin que los adultos tomen también la medida ('Calidad del aire de Bogotá tuvo niveles similares a los de Pekín' El Tiempo  4 de febrero de 2016). De esta manera con el datasets de registros RIPS del Hospital del Sur (Bogotá) a través de un proceso de ETL se puede determinar aquellos vinculados a IRA (Infecciones Respiratorias Agudas) y de esta manera relacionar los datos con datasets de otros orígenes como el RMCAB y SIVIGILA, sin olvidar que los datos se deben manejar de manera confidencial y segura cuando se realice la minería de datos.

Los datos que se manipularán en el proceso de ETL y minería de datos, se consideran sensibles y confidenciales, ya que es información personal y médica que en manos equivocadas puede causar inconvenientes a las personas vinculadas en los datos, de esta manera la arquitectura multi partita computacional, permitirá la comunicación entre los tres datasets para realizar la minería de datos sin que se filtre datos entre dichas bases de datos.

De esta manera poder aportar resultados importante para tomar medidas en la política de ambiente para la ciudad bogotana con enfoque a la salud de los habitantes. A pesar de la contaminación se podrá dar ampliar la visión de la problemática y los orígenes causados por la ubicación de la industria y movilidad de autos. Ademas 

\subsection{Objetivos}

\subsubsection{Objetivo General}
\begin{itemize}
\item Relacionar las enfermedades respiratorias con la contaminación del aire bogotano, mediante una arquitectura multi - partita computacional aplicando minería de datos.

\end{itemize}

\subsubsection{Objetivos Específicos}
\begin{itemize}
\item Determinar cuales son los datos relevantes requeridos para la investigación.
\item Realizar un proceso ETL (Extract Transform Load) a cada uno de los datasets y actualizar los datos a traves de una base de datos.
\item Diseñar la arquitectura multi - partita computacional flexible para aplicar minería de datos.
\item Plantear los algoritmos para realizar la minería de datos compatible a la arquitectura.
\item Aplicar los algoritmos de minería de datos a los datasets sin y con arquitectura multipartita computacional.
\item Interpretar los resultados dados en los procesos de datamining.
\item Comparar los resultados entre la minería aplicada en la arquitectura multi - partita contra los datasets sin arquitectura.

\end{itemize}

\subsection{Alcance y Limitaciones del Proyecto}
Dentro del proceso de vincular las bases de datos de manera anónima cada una independiente. Se relacionaran los datos relevantes de cada dataset para poder realizar un proceso de ETL, ya que la información inicial no es confiable y se encontrará desorganizada, de tal manera que el ETL se hará a través según la selección de las columnas para el diseño de una bodega de datos y poder realizar un filtro inicial de información. La estructura de la bodega, se dará a partir de cada dataset y el algoritmo que se determine después de entender las bases a través de diagramas de entidad relación.

El proceso de ETL se realizará con base a scripts realizados en Python, para poder exportar dichos datos en un archivo plano a una base de datos en Postergel de manera local y remota. Se diseñara un algoritmo flexible a realizar un proceso ETL express a información nueva que se vaya recibiendo de las fuentes públicas para ir aumentando los datos en la base.

Posteriormente se realizará un diseño a una arquitectura multi partita computacional, que permitirá comunicar las diferentes base de datos sin que exista el riesgo de fugas de datos entre ellas, ya que la información a analizar es confidencial y sensible, de tal manera que se debe preservar el anonimato. Para asegurar que no exista el riesgo de extracción de información, se realizará un proceso de encriptación de los datos a través de algoritmos de seguridad desarrollados en Python.

Ya implementada la arquitectura multi partita computacional, se utilizará la herramienta de RapidMiner para realizar el proceso de minería de datos, utilizando algoritmos de relación ya implementados en el software nombrado.

Los limites para el proceso de minería de datos serán dados por el volumen de datos que se logre filtra en el proceso de ETL, ya que las fuentes no siempre darán información confiable ni virgen para poder realizar un proceso ETL, de esta manera quedará sujeto al criterio de los investigadores de saber a que preguntas se desea encontrar respuesta en el proceso la minería de datos.

NOMBRAR TAMAÑO DE LOS DATASETS

\clearpage

\begin{center}
 \section{Marco de Referencia}
\end{center}

\subsection{Marco Teórico}
Para el desarrollo de este trabajo se definiran que tipos de datos y atributos seran tomados dentro de toda la información suministrada (Datasets) para el respectivo analisis de la información. En estos casos es de gran importancia definir que atributos van a ser la principal base para llegar a determinar factores de riesgo ambientales y de salud.En cierta manera toda esta serie de información sera importante para el correcto analisis y desarrollo del proyecto.
  
Es por esto que tanto los(RIPS,SIVIGILA,RMCAB) y demas entes que permitieron la recopilación de esta información seran de vital importancia para toda la investigación llevada a cabo.En cuanto a la contaminación en el aire como principal eje de este factor de riesgo ambiental es analizado tanto globalmentey que impactos tiene como localmente (Bogotá D.C)

Cabe afirmar que unos de los efectos que generan gran impacto a nivel local son  la combustión y efectos de cambio climatico, por esto  gran parte de la ciudad adelantan campañas con las cuales promueven la mejora de la calidad del aire que con el tiempo sera vital.






\subsection{Antecedentes o Estado del Arte}
La contaminación de aire en la ciudad de Bogotá ha sido de varios estudios por diferentes entes, ya que se relacionan directamente con enfermedades respiratorias de los habitantes de la capital, dichos estudios se han facilitado ya que desde 1997 la ciudad, cuenta con una red de monitoreo del aire (RCMAB) para la recolección y análisis de los datos recogidos. Las variables que se tienen en cuenta en la recolección son las meteorológicas y las concentraciones de los contaminantes para encontrar tendencias referente a la contaminación. Esta tecnología fue tomada de modelos ya implementados alrededor de mundo, por ejemplo se encuentran en Ciudad de México  (47 estaciones), Londres (3o estaciones) y Beijing (28 estaciones), las cuales suministra información importante para establecer políticas ambientales.

El sistema RCMAB utiliza para la medición sensores tipo DASIBI u OPSIS, los cuales medirán las dos tipos de variables que son las concentraciones atmosféricas y fenómenos meteorológicos, de las cuales se agrupan de la siguiente manera dada en el cuadro 1.0.

\begin{table}[htbp]
\begin{center}
\begin{tabular}{|l|l|}
\hline
Concentraciones Atmosféricas & Fenómenos  meteorológicos \\
\hline \hline
Óxidos de Nitrógeno & Precipitación \\ \hline
Dióxido de Azufre & Temperatura \\ \hline
Material Particulados en sus fracciones total & Radiación Solar \\ \hline
Material Particulados en sus fracciones Respirable & Velocidad del Viento \\ \hline
Material Particulados en sus fracciones Fina & Dirección del Viento \\ \hline
Ozono & Presión barométrica \\ \hline
Monóxido de Carbono & Húmedad Relativa \\ \hline
Metano &  \\ \hline
Benceno &  \\ \hline
Tolueno &  \\ \hline
Formaldehído &  \\ \hline
Hidrocarburos no metánicos & \\ \hline
\end{tabular}
\caption{Tabla Clasificación Variables RMCAB.}
\label{tabla:sencilla}
\end{center}
\end{table}


La ubicación de las estaciones se determino por la distribución de industrial de la ciudad, ya que la zona centro occidental se encuentran gran parte de las industrias y mayor movilización de autos en sus diferentes presentaciones. De esta manera con estudios previos se ha concluido que los valores límites establecidos son superados cada año ya que entre los años 1998 y 2005, siete de estas estaciones han reportado medias anuales que superan la norma anual para PM (Análisis del estado de la calidad del aire en Bogotá).

Para la vinculación de la contaminación dada en la ciudad de Bogotá, se debe tener en cuenta que es un causante determinante en las enfermedades respiratorias agudas ya que este tipo de infección es una de las principales causas de morbilidad y mortalidad (entre infantes menores a 5 años y personas de la tercera edad) (Influencia de la variabilidad climática en las enfermedades respiratorias agudas en Bogotá) con el agravante de ser una de las diez primeras causas de muerte en nuestro paìs (Influencia de la variabilidad climática en las enfermedades respiratorias agudas en Bogotá 
). De esta manera la búsqueda de relaciones directas entre las IRA y la información dada en los RMCAB será primordial para respaldar los estudios previos que se han hecho de manera individual. 

Igualmente con las variables que toma la RMCAB para realizar los análisis, se entiende que se recomienda el fortalecimiento del sistema para alertas tempranas enfocadas a la salud, por sugerencia del Intergovernmental Panel on Climate Change (IPCC) (Influencia de la variabilidad climática en las enfermedades respiratorias agudas en Bogotá). Ya que dentro de las polìticas que se deben establecer para el manejo de la contaminación ambiental referente a la salud pública, se basará a partir de los estudios realizados en los últimos años por la academia como fue en la Universidad de Los Andes donde se indafa sobre la influencia del clima en las IRA.



\subsection{Marco Legal}

La investigación se regirá bajo tres pilares que serán Salud Pública con énfasis en IRA, seguridad informática y regulación ambiental (calidad del aire) en la ciudad de Bogotá, de esta manera:
\begin{enumerate}[I]%for capital roman numbers.
\item\textbf{Salud Pública:}
	\begin{itemize}
    \item\textbf{Decreto 3518 de 2006 :} Por el cual se crea y reglamenta el 		Sistema de Vigilancia en Salud Pública y se dictan otras disposiciones.
	\item\textbf{Decreto 273 de 2004 :} Por la cual se crea el Comité Distrital 	para la Prevención y Atención de la Enfermedad Respiratoria Aguda y se dictan 	  otras disposiciones.
	\end{itemize}
\item\textbf{Seguridad Informática:}
	\begin{itemize}
    \item\textbf{Ley 74 de 1968: }  Incorpora a la legislación interna los Pactos 	  Internacionales de Derechos Económicos, Sociales y Culturales, de Derechos 	 Civiles de la ONU  
	\item\textbf{Ley 23, de 28 de enero de 1982: }que protege la imagen 		     individual frente a varias formas de abuso.
    \item\textbf{Ley 57 de 5 de junio de 1985: } Por la cual se ordena la 			publicidad de los actos y documentos oficiales.
    \item\textbf{Resolución 1995/1999 de 8 de junio 1999 del MPS: } Sobre manejo 	 de la historia clínica.
	\end{itemize}
\item\textbf{Calidad del Aire:}
	\begin{itemize}
    \item\textbf{Constitución Nacional Art. 79: }  Todas las personas tienen 		derecho a gozar de un ambiente sano. La Ley garantizará la participación de 	la comunidad en las decisiones que puedan afectarlo. Es deber del Estado 		proteger la diversidad e integridad del ambiente, conservar las áreas de 		especial importancia ecológica y fomentar la educación para el logro de estos 	  fines.
	\item\textbf{Ley 09 de 1979: }Código sanitario nacional.
    \item\textbf{Decreto 02 de 1982: } Reglamenta título I de la Ley 09-79 y el        decreto 2811-74. Disposiciones sanitarias sobre emisiones atmosféricas. Art. 	7 a 9 Definiciones y normas generales. Art.73 Obligación del Estado de 			mantener la calidad atmosférica para no causar molestias o daños que 			interfieran el desarrollo normal de especies y afecten los recursos 			naturales. Art. 74 Prohibiciones y restricciones a la descarga de material 		particulado, gases y vapores a la atmósfera. Art. 75 Prevención de la 			contaminación atmosférica.
    \item\textbf{Decreto 948 de 1995: } Normas para la protección y control de la 	   calidad del aire.
    \item\textbf{Resolución 1351 de 1995: } Se adopta la declaración denominada 	 Informe de Estado de Emisiones-IE1
    \item\textbf{Resolución 005 de 1996: } Reglamenta niveles permisibles de 		emisión de contaminantes por fuentes móviles.
    \item\textbf{Resolución 610 del 24 de marzo de 2010: } Con el fin de evaluar 	 el cumplimiento de los estándares de calidad de aire en Bogotá.
\end{itemize}
\end{enumerate}
\clearpage


\begin{center}
\section{Metodología}
\end{center}
Debe mostrar, en forma organizada, y precisa, cómo serán alcanzados cada uno de los objetivos
específicos propuestos. Deben detallarse, los procedimientos, técnicas, instrumentos, actividades,
etapas y demás estrategias metodológicas requeridas para el proyecto.  Tenga en cuenta que el
diseño metodológico es la base para planificar todas las actividades que demanda el proyecto y para
determinar los recursos humanos y financieros requeridos.

Un esquema útil puede ser:
\begin{enumerate}
  \item Un objetivo específico
  \begin{enumerate}
    \item se alcanza a través de unas etapas
    \begin{enumerate}
      \item que incluyen unos procedimientos
      \begin{itemize}
        \item que usan técnicas
        \item ... e instrumentos
      \end{itemize}
    \end{enumerate}
  \end{enumerate}
\end{enumerate}

Tuve la oportunidad de ver algunos trabajos radicados oficialmente, en esta parte dichos trabajos
vinculan las metodologías de desarrollo de software elegidas cuando son proyectos de desarrollo.
\clearpage

\begin{sidewaystable}
\begin{center}
\section{Cronograma}
\end{center}


\begin{center}

\begin{tabular}{l p{4cm} l p{4cm} l p{2cm} l p{2cm} l p{2cm} c p{2cm} c p{2cm} }
\toprule
\rowcolor[gray]{0.9}Objetivo específico & Descripción & Dependencias & Recursos  & Producto &
Inicio & Final \\
\midrule
Objetivo 0 & descripción 0 & depende de & recursos & se obtiene & semana 0 & semana 1 \\
\midrule
Objetivo 1 & descripción 1 & depende de & recursos & se obtiene & semana 2 & semana 3 \\
\bottomrule
\end{tabular}

\end{center}
\end{sidewaystable}
\clearpage

\begin{center}
\section{Recursos}
\end{center}
Se debe indicar el recurso humano, fisico, tecnológico necesario y disponible para llevar a cabo el
proyecto.
\clearpage

\begin{center}
\section{Cibergrafía}
\section{Anexos}
\end{center}
\clearpage



% Bibliografa.
%-----------------------------------------------------------------

\begin{thebibliography}{99}


\bibitem{Fogel05} Fogel, Karl.\emph{Producing Open Source Software, How to Run a
Successsful Free Software Project} Karl Fogel. 2005.

\bibitem{Huxley54} Huxley Aldous.\emph{The Doors of Perception}. Edhasa. 1954.

\end{thebibliography}


\end{document}